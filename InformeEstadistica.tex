\documentclass{article}

%paquetes necesarios
\usepackage{graphicx} %para incluir gráficos
\usepackage{amsmath} %para fórmulas matemáticas
\usepackage{booktabs} %para tablas
\usepackage{blindtext}
\usepackage{hyperref}
\usepackage[spanish]{babel}
%título del informe
\title{Informe sobre el dataset Data Science Salaries 2023}
\author{Borgo Martin, Sandoval Jose, Molina Leandro}

\hypersetup{
	colorlinks=true,
	linkcolor=blue,
	filecolor=magenta,      
	urlcolor=cyan,
	pdftitle={Overleaf Example},
	pdfpagemode=FullScreen,
}

\begin{document}
	
	%creación del título
	\maketitle
	\begin{abstract}
		El informe analiza el conjunto de datos de salarios de ciencia de datos del año 2023 de Kaggle, proporcionando un análisis detallado de las variables incluidas en el conjunto de datos. Se utilizó Python y Excel para la realización de gráficos y tablas. Este informe proporciona información valiosa para los profesionales de la ciencia de datos y los interesados en comprender el mercado actual.
	\end{abstract}
	\pagebreak
	\vspace{-20pt}
	
	\noindent
	\tableofcontents
	\pagebreak
	%introducción
	\section{Introducción}
	
	Aquí se debe describir el contexto del estudio, los objetivos y la metodología utilizada para el análisis de los datos.
	
	%resultados
	\section{Dataset Utilizado}
	\footnote{text}
	\footnote{text}
	
	\section{Análisis de variables y realización de gráficas}
\end{document}
